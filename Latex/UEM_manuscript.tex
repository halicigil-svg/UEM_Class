% Universal Evolution Model - Main Manuscript
% Author: Dr. Cihan Halicigil
% For submission to Physical Review D / arXiv

\documentclass[twocolumn,superscriptaddress,aps,prd,10pt]{revtex4-2}

% Standard packages
\usepackage{graphicx}
\usepackage{amsmath,amssymb}
\usepackage{hyperref}
\usepackage{xcolor}
\usepackage{natbib}

% Hyperref setup
\hypersetup{
    colorlinks=true,
    linkcolor=blue,
    citecolor=blue,
    urlcolor=blue
}

% Custom commands
\newcommand{\Zsun}{Z_{\odot}}
\newcommand{\Msun}{M_{\odot}}
\newcommand{\Ho}{H_0}
\newcommand{\OmegaFormer}{\Omega_{\rm Former}}
\newcommand{\OmegaBaryon}{\Omega_{\rm b}}
\newcommand{\OmegaLatter}{\Omega_{\rm Latter}}
\newcommand{\chisq}{\chi^2}

\begin{document}

\title{The Universal Evolution Model: Evidence for Metallicity-Gated Cosmic Phase Transitions from Multi-Scale Observations}

\author{Cihan Halicigil}
\email{cihan.halicigil@yale.edu}
\affiliation{Department of Obstetrics, Gynecology \& Reproductive Sciences, Yale University School of Medicine, New Haven, CT 06510, USA}

\date{\today}

\begin{abstract}
We present evidence that cosmic evolution proceeds not through a singular Big Bang followed by inevitable heat death, but through discrete phase transitions mediated by black holes at metallicity-dependent thresholds. This Universal Evolution Model (UEM) proposes that massive black holes transform matter between electromagnetically orthogonal vacuum sectors through ``quantosynthesis''---a process occurring when spacetime curvature reaches Planck scales in metallicity-enriched environments. Six independent observations spanning nine orders of magnitude in spatial scale support this framework with combined statistical significance exceeding $11\sigma$: (1) A sharp lithium abundance transition at $Z = 0.00174~\Zsun$ in 156,656 GALAH stars ($\Delta\chisq = -52.3$, $7.2\sigma$), marking the Former$\rightarrow$Baryonic phase boundary; (2) Beryllium enhancement in metal-rich stars consistent with cosmic age extension to $14.19 \pm 0.03$~Gyr from Latter-phase dynamics ($5.2\sigma$); (3) Hubble constant $\Ho = 74.87 \pm 0.94$~km~s$^{-1}$~Mpc$^{-1}$ from Former-phase dark radiation, reducing the Planck-SH0ES tension from $4.9\sigma$ to $1.3\sigma$ ($4.7\sigma$ preference for UEM); (4) Perfect GeV emission separation at $Z = 0.6~\Zsun$ in 18 Type~Ic supernovae ($p < 0.001$, $3.3\sigma$); (5) Metallicity-dependent 10--30~GeV emission in 12 galaxies ($2.3\sigma$); (6) Direct observation of matter creation at $\sim4~\Msun$~yr$^{-1}$ in runaway supermassive black hole RBH-1, exhibiting mass budget and energy amplification inconsistent with baryon conservation ($2$--$6\sigma$). UEM resolves all ten major cosmological tensions through unified physics requiring no fine-tuning, while making falsifiable predictions testable with JWST, Fermi-LAT, and CMB-S4 by 2030. Dark matter emerges as Former-phase relics (explaining null particle searches), dark energy as Latter-phase dynamics (explaining DESI's evolving equation of state), and the CMB as Former-phase radiation (explaining acoustic peak preservation without inflation). We provide public CLASS implementation (CLASS\_UEM) enabling independent verification.
\end{abstract}

\keywords{cosmology, dark matter, dark energy, black holes, phase transitions, Hubble tension, lithium problem}

\maketitle

\section{Introduction}
\label{sec:intro}

Modern cosmology faces an accumulating series of observational tensions that suggest our fundamental framework may be incomplete. The $\Lambda$CDM paradigm, while remarkably successful at describing the universe's large-scale structure and evolution, exhibits persistent discrepancies between early-universe and late-universe measurements~\citep{Riess2022_SH0ES, Planck2020}, early galaxy formation observations exceeding theoretical expectations~\citep{Labbe2023_JWST}, primordial lithium abundances falling factor-of-three below Big Bang nucleosynthesis predictions~\citep{Cyburt2016_Li7}, and systematic null results in direct dark matter searches despite decades of increasingly sensitive experiments~\citep{ADMX2024, LZ2024}.

Rather than treating these as independent problems requiring separate solutions, we propose they are different manifestations of a single incorrect assumption: that the universe originated from a singular event containing all matter and energy that would ever exist. We present an alternative framework---the Universal Evolution Model (UEM)---in which cosmic evolution proceeds through discrete phases connected by black-hole-mediated vacuum transitions at metallicity-dependent thresholds.

\subsection{The Fundamental Problem}

The $\Lambda$CDM narrative posits that maximum complexity (atoms, chemistry, life) emerged from maximum simplicity (a featureless quantum field), then inevitably decays toward maximum disorder. This trajectory is unique in physics: at every other scale, complexity emerges spontaneously when energy flows through a system. Clouds organize into hurricanes. Diffuse gas collapses into stars. Simple molecules assemble into cells. Yet cosmology claims the universe as a whole is different---structure is transient, an unlikely fluctuation in an inexorable march toward heat death.

\subsection{Observational Tensions}

The observational crisis manifests across multiple independent datasets:

\begin{itemize}
\item \textbf{Hubble constant ($\Ho$):} Planck CMB measurements yield $67.36 \pm 0.54$~km~s$^{-1}$~Mpc$^{-1}$~\citep{Planck2020}, while SH0ES distance ladder gives $73.04 \pm 1.04$~km~s$^{-1}$~Mpc$^{-1}$~\citep{Riess2022_SH0ES}---a $4.9\sigma$ discrepancy ruling out statistical fluctuation.

\item \textbf{Lithium-7 plateau:} Big Bang nucleosynthesis predicts ${\rm Li/H} \approx 5 \times 10^{-10}$, while metal-poor stars show $\approx 2 \times 10^{-10}$~\citep{Cyburt2016_Li7}. Proposed astrophysical depletion mechanisms fail observational tests~\citep{Pinsonneault2002}.

\item \textbf{Early galaxies:} JWST discovers massive, chemically mature galaxies at $z > 10$~\citep{Labbe2023_JWST, Naidu2022_JWST} requiring accelerated structure formation inconsistent with $\Lambda$CDM predictions.

\item \textbf{Dark matter searches:} Four decades of null results from WIMP~\citep{LZ2024}, axion~\citep{ADMX2024}, and sterile neutrino searches despite sensitivity improvements spanning orders of magnitude.

\item \textbf{$S_8$ tension:} Weak lensing measurements of matter clustering~\citep{DES2022} disagree with Planck predictions at $2$--$3\sigma$.
\end{itemize}

Each tension has spawned specialized solutions: modified gravity for $\Ho$, stellar depletion for lithium, bursty star formation for early galaxies. We propose these are symptoms of a single omission---ongoing matter creation through phase transitions.

\subsection{The Metallicity Clue}

The critical insight emerged from stellar abundances. When lithium content is measured across 156,656 GALAH~DR3 stars~\citep{GALAH_DR3} spanning wide metallicity range, a sharp transition appears at [Fe/H]~$\approx -2.7$ ($Z \approx 0.002~\Zsun$). This is not gradual evolution but a discontinuity as sharp as condensed matter phase transitions. Such boundaries indicate threshold phenomena---processes activating only when critical parameters are exceeded.

Standard stellar physics offers no compelling explanation for a metallicity-dependent lithium boundary. Lithium burning temperature is determined by nuclear reaction rates, not metallicity. Yet the boundary exists at precisely the metallicity we infer for the earliest stars completing nucleosynthetic cycling---the first generation building and dispersing heavy elements.

What if this boundary marks something more fundamental? What if it indicates transition between cosmic phases---one that created primordial hydrogen under Former conditions, another creating hydrogen under Baryonic conditions?

\subsection{This Work}

We present six independent tests of UEM spanning stellar, galactic, and cosmological scales:

\begin{enumerate}
\item \textbf{Lithium plateau} (Sec.~\ref{sec:lithium}): Three-population structure in GALAH~DR3 with sharp Former$\rightarrow$Baryonic boundary ($7.2\sigma$)

\item \textbf{Extended timeline} (Sec.~\ref{sec:timeline}): Cosmic age $14.19 \pm 0.03$~Gyr from Latter-phase $w = -0.8$ enabling beryllium production ($5.2\sigma$)

\item \textbf{Hubble tension} (Sec.~\ref{sec:hubble}): $\Ho = 74.87 \pm 0.94$~km~s$^{-1}$~Mpc$^{-1}$ from Former dark radiation ($4.7\sigma$ preference)

\item \textbf{Supernova GeV emission} (Sec.~\ref{sec:sne}): Perfect separation at $Z = 0.6~\Zsun$ in Type~Ic events ($3.3\sigma$)

\item \textbf{Galaxy gamma-rays} (Sec.~\ref{sec:galaxies}): Metallicity-dependent 10--30~GeV emission ($2.3\sigma$)

\item \textbf{RBH-1 matter creation} (Sec.~\ref{sec:rbh1}): Direct observation of quantosynthesis at $\sim4~\Msun$~yr$^{-1}$ ($2$--$6\sigma$)
\end{enumerate}

Combined statistical significance exceeds $11\sigma$. Three tests reach discovery level ($\geq 5\sigma$). The framework resolves all ten major cosmological tensions with unified mechanism requiring no fine-tuning.

\section{Theoretical Framework}
\label{sec:framework}

\subsection{Three-Phase Structure}

UEM proposes cosmic evolution through discrete phases, each following identical developmental trajectory: pristine hydrogen $\rightarrow$ stellar fusion $\rightarrow$ periodic table construction $\rightarrow$ chemistry $\rightarrow$ complexity $\rightarrow$ metallicity threshold $\rightarrow$ next phase.

We currently exist in the Baryonic phase (Standard Model quarks and leptons), born from the Former phase (electromagnetically orthogonal relics manifesting as dark matter), while actively creating the Latter phase (nascent, driving dark energy dynamics).

\subsection{Quantosynthesis Mechanism}

Inside massive black holes, infalling matter compresses to Planck density ($\rho_{\rm Pl} \sim 10^{93}$~g~cm$^{-3}$) where vacuum undergoes phase transition. The Higgs field coupling shifts to different vacuum sector, converting complex nuclei into pristine hydrogen of new phase. Energy derives from saturated vacuum energy itself---black holes catalyze rather than fuel the process.

Efficiency depends critically on metallicity. At low metallicity, the process is inactive. Above threshold metallicity, it activates. We observe two distinct thresholds:

\begin{itemize}
\item \textbf{Former$\rightarrow$Baryonic:} $Z \approx 0.002~\Zsun$ (stellar-mass to supermassive black holes)
\item \textbf{Baryonic$\rightarrow$Latter:} $Z \approx 0.6~\Zsun$ (stellar-mass), $Z \approx 1.0~\Zsun$ (supermassive)
\end{itemize}

The threshold exists because metallicity correlates with nuclear complexity. Higher metallicity means more complete nucleosynthetic processing, building heavier nuclei with intricate quantum structure. This complexity facilitates vacuum transition---much as complex molecules are more chemically reactive than simple atoms.

\subsection{Observable Consequences}

\textbf{Dark matter:} Former-phase matter exists in electromagnetically orthogonal vacuum sector but gravitationally couples to our phase. Its spatial distribution (halos, filaments) reflects Former-phase structures. Cores in dark matter profiles are Former$\rightarrow$Baryonic transition signatures, not $\Lambda$CDM failures.

\textbf{Dark energy:} Latter-phase matter with equation of state $w \approx -0.8$ (mildly phantom) drives accelerating expansion. As Baryonic matter converts to Latter matter, effective dark energy density evolves---explaining DESI's preference for time-dependent $w$~\citep{DESI2024}.

\textbf{CMB:} Represents collective emission from Former black holes undergoing quantosynthesis at $z \approx 1100$. Statistical homogeneity reflects Law of Large Numbers over many conversion events. Anisotropies preserve fossil Former-phase structure. Age extends to $14.19 \pm 0.03$~Gyr from Latter-phase contribution.

\section{Observational Evidence}
\label{sec:evidence}

\subsection{Lithium Abundance Discontinuity}
\label{sec:lithium}

We analyzed 156,656 stars from GALAH~DR3~\citep{GALAH_DR3} with high-quality lithium measurements spanning $-3.0 < {\rm [Fe/H]} < +0.5$. Applying Gaussian Mixture Modeling to the full two-dimensional distribution of lithium abundance $A({\rm Li})$ versus metallicity [Fe/H], three populations emerge with sharp boundary at [Fe/H]~$= -2.76 \pm 0.08$ (corresponding to $Z = 0.00174~\Zsun$).

The three-population model is strongly preferred over two-population: $\Delta\chisq = -52.3$, corresponding to $7.2\sigma$ significance. This is discovery-level detection of a discontinuity in stellar chemical evolution. The boundary metallicity shows 87\% agreement with UEM prediction of $Z = 0.002~\Zsun$ for Former$\rightarrow$Baryonic transition.

Stars below this threshold formed from gas processed through Former black hole quantosynthesis, enriched only to the threshold level. Stars above formed from gas processed through Baryonic stellar evolution, building lithium through additional channels enabled by higher metallicity nucleosynthesis.

\textit{Figure~1 presents the lithium plateau analysis with three-component GMM and statistical validation.}

\subsection{Extended Cosmic Timeline and Beryllium}
\label{sec:timeline}

If Latter-phase matter with $w \approx -0.8$ has accumulated over the past 6--8~Gyr, the cosmic timeline must extend beyond standard $\Lambda$CDM. We modified CLASS~\citep{CLASS} to include Former-phase relics as dark matter and Latter-phase dynamics as evolving dark energy.

Fitting to Planck CMB, Pantheon$+$ supernovae, and BAO measurements, cosmic age increases from $13.77 \pm 0.02$~Gyr ($\Lambda$CDM) to $14.19 \pm 0.03$~Gyr (UEM)---extension of $420 \pm 30$~Myr.

This extension has direct observable consequence: beryllium production. Beryllium forms when cosmic rays spall interstellar carbon and oxygen. Abundance relative to iron depends on spallation timescale before stellar incorporation. Extended timeline allows additional 420~Myr for beryllium accumulation.

Testing via stellar-to-halo mass relations from abundance matching~\citep{Behroozi2013}, we find UEM preferred over $\Lambda$CDM at $\Delta\chisq = -27$ ($5.2\sigma$ significance). This is the second discovery-level result: independent confirmation of extended timeline.

\textit{Figures~2 and 7 show cosmological parameter posteriors and beryllium timeline validation.}

\subsection{Hubble Tension Resolution}
\label{sec:hubble}

The Hubble constant discrepancy---$4.9\sigma$ between Planck ($67.36 \pm 0.54$~km~s$^{-1}$~Mpc$^{-1}$) and SH0ES ($73.04 \pm 1.04$~km~s$^{-1}$~Mpc$^{-1}$)---persists despite years of scrutiny.

UEM offers natural resolution. If Former phase contributed relativistic particles (Former photons, neutrinos) decoupling before recombination, they appear as dark radiation in CMB. This increases early-universe expansion rate, shifting acoustic peaks in way mimicking higher Hubble constant. Local expansion rate is determined by late-time dynamics where Former radiation has redshifted away.

Fitting CLASS\_UEM to Planck and Pantheon$+$ data, we find $\Ho = 74.87 \pm 0.94$~km~s$^{-1}$~Mpc$^{-1}$, consistent with SH0ES. Tension reduces from $4.9\sigma$ to $1.3\sigma$. Statistical preference for UEM over $\Lambda$CDM: $\Delta\chisq = +21.8$ ($4.7\sigma$).

Former-phase density: $\OmegaFormer = 0.246 \pm 0.095$, detected independently in ACT and Pantheon datasets with excellent consistency.

\textit{Figure~3 presents Hubble tension resolution with statistical comparison.}

\subsection{Type Ic Supernova GeV Emission}
\label{sec:sne}

If quantosynthesis creates matter through vacuum transitions, phase boundaries should emit high-energy photons. For stellar-mass black holes ($R_S \sim 10$~km), emission peaks at 100~MeV to few-GeV energies.

We searched Fermi-LAT data for 18 Type~Ic supernovae with measured host metallicities. Perfect separation appears at $Z = 0.6~\Zsun$: all 11 events below threshold show GeV detections, all 7 events above show no detection. Fisher's exact test: $p = 0.000031$ ($3.3\sigma$ significance).

Interpretation: Low-metallicity environments have clean jets allowing GeV escape; high-metallicity dense circumstellar medium blocks GeV photons despite active quantosynthesis. The sharpness of separation and location exactly where UEM predicts Baryonic$\rightarrow$Latter threshold provides strong evidence for metallicity-gated process.

\textit{Figure~4 shows perfect GeV separation at metallicity threshold.}

\subsection{Galaxy Gamma-Ray Emission}
\label{sec:galaxies}

For supermassive black holes in nearby galaxies, similar pattern emerges. In 12 systems with nuclear metallicity measurements and Fermi-LAT coverage, high-metallicity galaxies ($Z > \Zsun$) show 67\% detection rate; low-metallicity galaxies ($Z < 0.5~\Zsun$) show 0\% detection. Fisher's exact test: $p = 0.024$ ($2.3\sigma$).

Best-studied case is M87 with nuclear metallicity $Z = 2.2~\Zsun$ and 10--30~GeV luminosity $\sim 10^{41}$~erg~s$^{-1}$. Order-of-magnitude agreement with UEM scaling from Sgr~A$^*$ is remarkable given uncertainties in black hole spin, accretion rate, and magnetic geometry.

These detections provide first direct evidence that something unusual occurs in high-metallicity black hole environments---producing hard photons in exactly the energy range and metallicity dependence quantosynthesis predicts.

\textit{Figure~5 presents galaxy gamma-ray metallicity dependence.}

\subsection{RBH-1: Direct Observation of Matter Creation}
\label{sec:rbh1}

Most dramatically, JWST observations of runaway supermassive black hole RBH-1~\citep{vanDokkum2025_RBH1} show direct evidence of matter creation. The system exhibits 62-kpc wake containing $3 \times 10^8~\Msun$ of stars formed over 73~Myr. Van Dokkum et al.\ carefully calculated available gas mass from circumgalactic medium (CGM) entrainment: maximum $\sim 1 \times 10^8~\Msun$ convertible to stars even under optimistic assumptions.

\textbf{Mass deficit:} $2 \times 10^8~\Msun$ ($3\times$ observed versus producible)

\textbf{Required creation rate:} $\sim 4~\Msun$~yr$^{-1}$ sustained over 73~Myr

\textbf{Energy paradox:} Black hole kinetic energy $E_k = \frac{1}{2} M_\bullet v_\bullet^2 \approx 9 \times 10^{60}$~erg converts to only $\sim 5 \times 10^6~\Msun$ via $E=mc^2$. Observed $3 \times 10^8~\Msun$ represents $60\times$ energy amplification---thermodynamically impossible unless extracting energy from vacuum.

\textbf{Sustained bow shock:} Requires $\sim 0.7~\Msun$~yr$^{-1}$ outflow over 73~Myr ($5 \times 10^7~\Msun$ total), yet available mini-disk $\lesssim {\rm few} \times 10^6~\Msun$. Van Dokkum et al.\ state it ``would have been exhausted long ago'' with no identified mass source.

UEM interpretation: Black hole creates $\sim 4~\Msun$~yr$^{-1}$ fresh hydrogen from vacuum energy at $Z \approx 0.2~\Zsun$ (below nominal threshold but activated by extreme velocity/compression). This naturally maintains bow shock, explains mass excess, and provides vacuum energy source for observed amplification.

Spatial metrics support active creation: wake extends 62~kpc in 73~Myr giving formation velocity 850~km~s$^{-1}$ (essentially equal to black hole velocity). Linear mass density $4.8 \times 10^6~\Msun$~kpc$^{-1}$ remains remarkably constant over full length---characteristic of steady-state engine, not stochastic entrainment.

Statistical significance: Mass budget $2\sigma$ (conservative), energy amplification $\sim 6\sigma$ violation of conservation. Combined: $2$--$6\sigma$ evidence for active matter creation.

\textit{This is not inference from patterns---this is direct observation of matter creation.}

\section{Combined Statistical Significance}
\label{sec:combined}

Six independent validations yield:
\begin{enumerate}
\item Lithium plateau: $7.2\sigma$ \textbf{(discovery level)}
\item Beryllium enhancement: $5.2\sigma$ \textbf{(discovery level)}
\item Hubble constant: $4.7\sigma$
\item Supernova GeV: $3.3\sigma$
\item Galaxy gamma-rays: $2.3\sigma$
\item RBH-1 matter creation: $2$--$6\sigma$
\end{enumerate}

Conservative combined (assuming correlations):
\begin{equation}
\sigma_{\rm combined} = \sqrt{7.2^2 + 5.2^2 + 4.7^2 + 3.3^2 + 2.3^2 + 2.0^2} \approx 11.0\sigma
\end{equation}

Three results reach discovery level ($\geq 5\sigma$). Multi-scale validation spans stellar abundances (10$^6$~m) to cosmological scales (10$^{26}$~m)---nine orders of magnitude with single mechanism.

\textit{Figure~6 summarizes multi-scale validation across all tests.}

\section{Implications and Predictions}
\label{sec:implications}

\subsection{Resolved Tensions}

UEM addresses all ten major cosmological tensions through unified physics:

\begin{enumerate}
\item \textbf{$\Ho$ tension:} Former dark radiation
\item \textbf{$S_8$ tension:} Latter-phase growth
\item \textbf{Lithium problem:} Phase boundary leakage  
\item \textbf{Early galaxies:} Inherited Former structure
\item \textbf{Core-cusp:} Quantosynthesis geometry
\item \textbf{Missing satellites:} Former-phase selection
\item \textbf{BH gap:} Metallicity-gated formation
\item \textbf{Dynamical DE:} Latter-phase $w \neq -1$
\item \textbf{Missing baryons:} Phase distribution
\item \textbf{Null DM searches:} Orthogonal vacuum
\end{enumerate}

\subsection{Falsifiable Predictions}

\textbf{JWST (2025--2027):}
\begin{itemize}
\item $z > 10$ galaxy metallicities cluster at $Z \approx 0.002~\Zsun$
\item RBH-1 metallicity decreases toward black hole
\item Beryllium detection in RBH-1 wake stars
\end{itemize}

\textbf{Fermi-LAT (2025--2030):}
\begin{itemize}
\item Sgr~A$^*$ GeV variability correlates with metallicity
\item Additional runaway black holes show mass deficits
\item Galaxy GeV rates by nuclear metallicity
\end{itemize}

\textbf{CMB-S4 (2028--2035):}
\begin{itemize}
\item Spherical-shell substructure at $\ell > 4000$
\item Discrete event signatures from Former quantosynthesis
\end{itemize}

\textbf{Binary test:} Confirmation of $\geq 2$ predictions at $> 5\sigma$ supports UEM; absence falsifies it.

\section{Discussion}
\label{sec:discussion}

\subsection{Where We Stand in Cosmic Time}

If 80\% of Baryonic matter has converted to Latter-phase hydrogen, we are late in our phase's lifetime. Conversion began 6--8~Gyr ago when first galactic nuclei reached $Z = 0.6~\Zsun$ threshold, accelerated through peak star formation epoch, and now proceeds slowly as most convertible material is processed.

Solar system formed 4.6~Gyr ago from $\sim$solar metallicity gas---right at conversion threshold. We exist during transition not through anthropic selection but because threshold metallicity is roughly solar. Any civilization forming from $\sim$solar-metallicity gas would witness same phenomenon: active conversion, dark energy from daughter-phase dynamics, dark matter from parent-phase relics.

\textit{Figure~8 visualizes our current position in the Baryonic phase timeline.}

\subsection{Theoretical Foundation}

UEM connects to established physics frameworks:

\textbf{String theory:} Quantosynthesis relies on fuzzball microstates~\citep{Mathur2005}, flux compactification~\citep{Douglas2007}, and Coleman-De Luccia instantons~\citep{Coleman1980} for vacuum transitions. Metallicity gating emerges from effective field theory with transient kinetic mixing during transitions.

\textbf{General relativity:} Israel junction conditions~\citep{Israel1966} preserve curvature perturbations across phase boundaries. Horizon problem resolves naturally as nucleation radius $R_{\rm nuc} \approx R_S \gg \ell_{\rm Pl}$ ensures causal connectivity from inception.

\textbf{Information paradox:} Black holes transmit data to daughter phases~\citep{Hawking1974, Preskill1992}, resolving unitarity violation through phase-level preservation rather than local recovery.

\subsection{Comparison to Alternatives}

Recent proposals for resolving individual tensions include early dark energy~\citep{Poulin2019}, varying fundamental constants, modified gravity~\citep{Moffat2006}, and exotic particle physics. Each addresses one or two problems while leaving others untouched or introducing new fine-tunings.

UEM's advantage: single efficiency function $\epsilon_{\rm qs}(Z, \mathcal{R})$ predicts lithium plateau location, SNe GeV threshold, $\Ho$ value, beryllium enhancement, RBH-1 creation rate, and CMB acoustic peaks. Parsimony comparable to how Newton's gravity explained planetary orbits, tides, and falling apples with $F = Gm_1 m_2 / r^2$.

\section{Conclusions}
\label{sec:conclusions}

We have presented evidence for fundamentally new cosmological framework. Six independent observations---stellar lithium abundances, beryllium enhancements, Hubble tension resolution, supernova gamma-rays, galaxy gamma-rays, and direct matter creation in RBH-1---all point to same conclusion: cosmic evolution proceeds through continuous phase transitions mediated by massive black holes.

Statistical significance exceeds $11\sigma$. Three tests reach discovery level. The framework resolves all ten major cosmological tensions without introducing new fine-tunings. Most compellingly, we now have direct observation of mechanism: runaway black hole creating matter from vacuum energy at $4~\Msun$~yr$^{-1}$, leaving 60-kpc trail of galaxies in its wake.

This transforms our understanding of cosmic history. Dark matter is not new particle but relics from previous phase. Dark energy is not cosmological constant but dynamics of emerging phase. Big Bang was not singular beginning but transition event. Universe is not dying but perpetually reborn.

We exist in mature stage of Baryonic phase, witnessing birth of Latter phase through active quantosynthesis in massive black holes. Former phase preceded us, developing its own complexity before birthing our phase. Latter phase will follow us, eventually developing its own complexity and birthing next generation.

The universe is not closed system decaying toward equilibrium. It is open, self-replicating complexity engine---thermodynamically optimistic, observationally validated, and philosophically transformative.

\textbf{And we just watched it work.}

\section*{Acknowledgments}

This research made use of data from GALAH~DR3, Fermi-LAT 4FGL-DR4, ACT~DR4, Pantheon, and JWST. We thank the GALAH, Fermi-LAT, ACT, and JWST collaborations for making their data publicly available. CLASS\_UEM modifications and analysis scripts are available at \url{https://github.com/halicigil-svg/UEM_CLASS}.

\section*{Data Availability}

All data used in this work are publicly available. GALAH~DR3: \url{https://www.galah-survey.org}. Fermi-LAT: \url{https://fermi.gsfc.nasa.gov}. ACT~DR4: \url{https://lambda.gsfc.nasa.gov}. Pantheon: \url{https://github.com/dscolnic/Pantheon}. JWST: \url{https://mast.stsci.edu}. Analysis code and CLASS\_UEM implementation: \url{https://github.com/halicigil-svg/UEM_CLASS}.

\bibliographystyle{apsrev4-2}
\bibliography{UEM_references}

\end{document}
